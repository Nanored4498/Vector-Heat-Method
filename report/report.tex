\documentclass[11pt]{article}

\usepackage[french]{babel}
\usepackage[utf8x]{inputenc}  
\usepackage[T1]{fontenc}
\usepackage[left=2.7cm,right=2.7cm,top=3cm,bottom=3cm]{geometry}
\usepackage{amsmath,amssymb,amsfonts}
\usepackage{kpfonts}
\usepackage{tikz}
\usepackage{bbm}
\usepackage{hyperref}

\newcommand{\trans}{\mathsf{T}}
\newcommand{\syst}[2]{\left\{ \begin{array}{#1} #2 \end{array} \right.}
\newcommand{\pmat}[1]{\begin{pmatrix} #1 \end{pmatrix}}
\newcommand{\R}{\mathbb{R}}
\newcommand{\N}{\mathbb{N}}

\title{
	\noindent\rule{\linewidth}{0.4pt}
	{ \huge Modèles Déformables et Méthodes Géodésiques en Analyse d’images } \\
	Projet : The Vector Heat Method
	\noindent\rule{\linewidth}{1pt}
}

\author{Yoann Coudert-\,-Osmont}

\begin{document}
	
	\maketitle
	
	\section{Etude synthétique de l'article}
	
	Environ 2 pages
	
	\paragraph{Problème traité}
	Le papier propose une méthode pour transporter parallèlement un vecteur sur une variété.
	
	\paragraph{Equations et méthodes numériques utilisées}
	
	\paragraph{Comparaison avec le cours}
	
	\paragraph{Originalité (selon les auteurs)}
	
	\paragraph{Résultats nouveaux}
	
	\paragraph{Faiblesses}
	
	\section{Résumé}
	
	Environ 3 pages

\end{document}