\documentclass[11pt]{article}

\usepackage[french]{babel}
\usepackage[utf8x]{inputenc}  
\usepackage[T1]{fontenc}
\usepackage[left=2.7cm,right=2.7cm,top=3cm,bottom=3cm]{geometry}
\usepackage{amsmath,amssymb,amsfonts}
\usepackage{kpfonts}
\usepackage{tikz}
\usepackage{bbm}
\usepackage{hyperref}
\usepackage{algorithm}
\usepackage{algpseudocode}

\newcommand{\trans}{\mathsf{T}}
\newcommand{\syst}[2]{\left\{ \begin{array}{#1} #2 \end{array} \right.}
\newcommand{\pmat}[1]{\begin{pmatrix} #1 \end{pmatrix}}
\newcommand{\R}{\mathbb{R}}
\newcommand{\N}{\mathbb{N}}

\DeclareMathOperator{\trace}{tr}
\DeclareMathOperator{\grad}{grad}
\DeclareMathOperator{\divergence}{div}
\DeclareMathOperator{\connexionLaplacian}{\Delta^{\hspace{-3pt} \nabla}}
\DeclareMathOperator*{\argmin}{arg\,min}


\title{
	\noindent\rule{\linewidth}{0.4pt}
	{ \huge Modèles Déformables et Méthodes Géodésiques en Analyse d’images } \\
	Projet : The Vector Heat Method \cite{VHM}
	\noindent\rule{\linewidth}{1pt}
}

\author{Yoann Coudert-\,-Osmont}

\begin{document}
	
	\maketitle
	
	\section{Etude synthétique de l'article}
	
	\emph{Environ 2 pages} \\
	
	\paragraph{Problème traité}
	Le papier propose une méthode pour transporter parallèlement un vecteur selon les géodésiques d'une variété. Cette méthode donne par la suite, la possibilité de calculer le $\log$ riemannien de manière efficace, de calculer des moyennes de Fréchet, ou encore de construire des diagrammes de Voronoï.
	
	\paragraph{Equations et méthodes numériques utilisées}
	L'équation de la chaleur est utilisée pour calculer les distances géodésiques. Comme on a pu le voir en TP, cette méthode est très efficace et converge sous certaines conditions vers les vrais distances géodésiques. Ainsi les géodésiques ne sont pas calculées explicitement. Ces calculs généralement complexes sont remplacés par de simples résolutions de systèmes linéaires creux. Ces systèmes sont définis avec un Laplacien qui est défini sur beaucoup de types de données différentes (maillages, nuages de points, voxels ...) ce qui en fait un point fort de ce papier, l'adaptabilité.
	
	\paragraph{Comparaison avec le cours}
	On retrouve l'équation de la chaleur que nous avons utilisé dans un TP pour calculer des distances géodésiques et pour échantillonner un maillage à faces triangulaires.
	
	\paragraph{Originalité (selon les auteurs)}
	Ce papier est dans la continuité d'un précédent papier de Keenan Crane, sur le calcul des distances géodésiques via l'équation de la chaleur \cite{HM}. Leur approche est original dans le sens où ils semblent être les seuls à avoir creuser cette approche par l'équation de la chaleur pour le transport parallèle. Les approches précédentes se basaient généralement sur un calcul explicite des géodésiques. Or calculer toutes les distances à partir d'un sommet de manière exacte a une complexité en $\mathcal{O}(n^2 \log n)$, où $n$ est le nombre d'arrêtes. Cette complexité est bien trop grande et les algorithmes qui approximent ces géodésiques à moindre coût peuvent donner des résultats trop différents. Ici la solution proposé possède une complexité proche de $\mathcal{O}(n)$.
	
	\paragraph{Résultats nouveaux}
	Selon les auteurs, ils sont les premiers à proposé un calcul de la carte logarithme efficace. Selon une figure donnée dans l'article les anciennes méthodes engendraient des erreurs non négligeables sur l'orientation pour les sommets assez éloignés du sommet initial. Ce résultats entraîne immédiatement une première méthode de calcul de moyennes de Fréchet précise et efficace.
	
	\paragraph{Faiblesses}
	
	\section{Résumé}
	
	\emph{Environ 3 pages} \\
	
	\subsection{Source unique}
	Pour un champs scalaire $\phi_t$ qui évolue au cours du temps on rappelle que l'équation de la chaleur s'écrit :
	$$ \frac{d}{dt} \phi_t = \Delta \phi_t $$
	Où $\Delta$ est l'opérateur Laplacien défini comme la divergence du gradient ou encore, comme la trace de la Hessienne :
	$$ \Delta f = \trace \left( H(f) \right) = \divergence \circ \grad f $$
	Pour un champs de vecteur on peut définir une équation de la chaleur de manière similaire. On remplace simplement le gradient par une dérivée covariante/connexion. Et on remplace les opérateurs qui découlent directement du gradient par les opérateurs qui découlent directement d'une connexion. Le papier propose naturellement la connexion de Levi-Civita $\nabla$. On peut alors définir un Laplacien de connexion $\connexionLaplacian$ par :
	$$ \connexionLaplacian X = \trace \left( \nabla^2 X \right) = - \nabla^* \nabla X $$
	Où la dérivée covariante seconde est définie comme suit :
	$$ \nabla^2_{X, Y} Z = \nabla_X \nabla_Y Z - \nabla_{\nabla_X Y} Z $$
	Et où $\nabla^*$ est l'adjoint de la connexion.
	En prenant pour champs de vecteur au temps zéro, $X_0 = \delta_x v$ le champs valant $v$ en $x$ et nul partout ailleurs, la solution de l'équation de la chaleur vectorielle s'écrit alors :
	$$ k_t^\nabla(x, y) \cdot v = \frac{e^{-d(x, y)^2 / 4t}}{(4 \pi t)^{n/2}} j(x, y)^{-1/2} \left( \sum_{i=0}^\infty t^i \Psi_i(x, y) \right) \cdot v $$
	Où la première fonction de la série est l'opérateur de transport parallèle le long de la géodésique minimale entre $x$ et $y$ :
	$$ \Psi_0(x, y) = P_{\gamma_{x \rightarrow y}} $$
	En ce sens, lorsque $t$ tend vers 0, la solution se trouve être le transport parallèle de $v$ selon les géodésiques à une constante multiplicative près variant selon la position $y$. Cette constante multiplicative est la solution de l'équation de la chaleur scalaire partant d'un champs initial $\phi_0 = \delta_x |v|$ valant la norme de $v$ en $x$ et étant nul partout ailleurs. On note cette solution $k_t(x, y)$, et on obtient :
	$$ \lim_{t \rightarrow 0} \frac{k_t^\nabla(x, y)}{k_t(x, y)} = P_{\gamma_{x \rightarrow y}} $$
	
	\subsection{Multiples sources}
	
	Dans le cas où notre champ initial n'est plus un simple Dirac mais une partie $\Omega \subset M$ de notre variété, on souhaite que le vecteur obtenu au point $q$ soit le transporté du vecteur au point $p$ de $\Omega$ le plus proche. On note alors $X$ le champs de vecteur initial nul sur $M \setminus \Omega$, et $\bar{X}$ le champ obtenu par transport parallèle. On utilisera la notation $X|_p$ pour la valeur du champ $X$ au point $p$. On souhaite obtenir le champ suivant :
	$$ \bar{X}|_q = \syst{ll}{
		X|_q & \text{Si } q \in \Omega \\
		P_{\gamma_{p \rightarrow q}} \cdot X|_p & \text{Si } q \notin \Omega \text{ et } p = \argmin_{r \in \Omega} d(q, r)
	} $$
	Soit alors deux scalaires distincts $u$ et $v$ et deux positions $x$ et $y$ sur $M$. On remarque la limite suivante :
	$$ \lim_{t \rightarrow 0} \frac{u k_t(x, z) + v k_t(y, z)}{k_t(x, z) + k_t(y, z)} = \syst{ll}{
		u & \text{Si } d(x, z) < d(y, z) \\
		v & \text{Si } d(x, z) > d(y, z) \\
		(u+v) / 2 & \text{Si } d(x, z) = d(y, z) \\
	} $$
	Le numérateur est la solution de la diffusion de $\delta_x u + \delta_y v$ alors que le dénominateur est la solution de $\delta_x + \delta_y$. La fraction constituée retourne ainsi la valeur associée point le plus proche. On peut étendre cette idée à un domaine $\Omega$ quelconque afin de renormaliser correctement notre champ vectorielle après diffusion. Il suffit alors de considéré les solutions des équations suivantes :
	$$ \frac{d}{dt} u_t = \Delta u_t \qquad \text{avec } \quad u_0 = | X | $$
	$$ \frac{d}{dt} \phi_t = \Delta \phi_t \qquad \text{avec } \quad \phi_0 = \mathbbm{1}_\Omega $$
	Puis on définira la norme de $\bar{X}$ via :
	$$ | \bar{X} | = u_t / \phi_t $$
	La direction de $\bar{X}$ en revanche sera celle obtenue via la résolution de l'équation de la chaleur au temps $t$ partant de la condition initiale $\bar{X}_0 = X$. Cela permet alors de définir l'algorithme \ref{alg:VHM} pour calculer le transport parallèle d'un champ de vecteur.
	\begin{algorithm}[h]
		\caption{Vector Heat Method}
		\label{alg:VHM}
		\begin{algorithmic}
			\State \textbf{Entrée:} Un champ de vecteur $X$ sur une partie $\Omega$ d'une variété $M$.
			\State \textbf{Sortie:} Un champ de vecteur $\bar{X}$ sur $M$ tout entier.
			\State \hspace{10pt} I. \; Résoudre $\frac{d}{dt} Y_t = \connexionLaplacian Y_t$ au temps $t$ avec la condition initiale $Y_0 = X$.
			\State \hspace{6pt} II. \; Résoudre $\frac{d}{dt} u_t = \Delta u_t$ au temps $t$ avec la condition initiale $u_0 = |X|$.
			\State \, III. \; Résoudre $\frac{d}{dt} \phi_t = \Delta \phi_t$ au temps $t$ avec la condition initiale $\phi_0 = \mathbbm{1}_\Omega$.
			\State \hspace{3pt} IV. \; Construire le champ de vecteur $\bar{X}_t = ( u_t Y_t) / (\phi_t |Y_t|)$.
		\end{algorithmic}
	\end{algorithm}
	
	\appendix
	
	\bibliographystyle{alpha}
	\bibliography{bib.bib}
	
	\section{Compléments de géométrie différentielle}
	
	Cette section a pour but de clarifier certaines notions non abordées dans les cours qui traitent de géométrie différentielle au sein du master MVA. On utilisera la convention de sommation d'Einstein.
	
	\subsection{Seconde dérivée covariante}
	
	On souhaite définir un opérateur linéaire de second ordre $\nabla^2$ à l'aide d'une dérivée covariante $\nabla$. Soit $X, Y$ et $Z$ des champs de vecteurs. Par linéarité on doit avoir.
	$$ \nabla_{X, Y}^2 Z = X^i Y^j \nabla_{i, j}^2 Z = X^i Y^j \nabla_i \left( \nabla_j Z \right) $$
	Or on a aussi :
	$$ \nabla_X \left( \nabla_Y Z \right) = X^i \nabla_i \left( Y^j \nabla_j Z \right) = X^i Y^j \nabla_i \left( \nabla_j Z \right) + X^i \left( \nabla_i Y^j \right) \left( \nabla_j Z \right) = \nabla_{X, Y}^2 Z + \left( \nabla_X Y^j \right) \left( \nabla_j Z \right) $$

\end{document}